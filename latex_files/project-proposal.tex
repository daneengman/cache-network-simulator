\documentclass{article}
\usepackage{graphicx} % Required for inserting images
\usepackage{booktabs}



\title{15-418 Final Project Proposal}

\author{Yu-Ching Wu (yuchingw) and Dane Engman (dengman)}
\date{\today}

\begin{document}

\maketitle

\section{Website URL}
https://daneengman.github.io/cache-network-simulator/

\section{Summary}
We plan to augment Professor Railing's Computer Architecture Design Simulator for Students by developing a cycle-accurate network topology simulator that can simulate how cache coherency messages would propagate about the network. We plan to use this We will write the network simulator in SystemVerilog and integrate it into the existing
        C++ framework by using the SystemVerilog DPI.

\section{Background}
The network topology of a cache interconnect is an important microarchitectual feature that will influence performance during various cache transactions, but the exact ways that different topologies might affect performance is unclear.
The CADSS is a lightweight architecture simulator that will allow us to easily integrate our network simulator without facing the complexities of a more complex simulator like Gem 5. 

SystemVerilog is a hardware description and verification language which we will use to describe the network. It implicitly supports constructs of timing and and parallelism that are not present in software languages, and will allow us to integrate our simulation with other tools that can measure power, area and other requirements. 

\section{Challenges}
We will need to learn how to use the SystemVerilog DPI to interact with the C++ simulator and accurately reflect performance benefits and trade offs. 

We will also need to develop a variety of network topologies to test performance with. Given that we are working with a HDL we will need to handle correctness issues and a level of parallelism that is very high if we want to accurately represent high performance topologies. 

\section{Resources}
As noted before, we will use the CADSS along with SystemVerilog and its DPI. We additionally hope to use some of Professor Railing's traces to test the performance of different designs. 

\section{Goals and Deliverables}
\subsection{Plan to Achieve}
We expect to achieve:
\begin{enumerate}
    \item Design and integrate interface between the CADSS and an arbitrary network for sending cache-coherence messages
    \item Test design with simple crossbar network with no latency 
    \item Design at least 3 different networks (Bus, ring, torus) and compare performance between different designs
\end{enumerate}
\subsection{Hope to Achieve}
We hope to achieve:
\begin{enumerate}
    \item More advanced networks such as bufferless networks, "fat trees" with greater bandwidth near root, multi-stage networks
    \item Perform energy and power analysis of different topologies using synthesis tools
    \item Try different cache coherence methodologies such as MESI, MOESI to see how different interconnects work with different workload coherence management patterns
\end{enumerate}

\subsection{Plans for Demonstration}

We hope to present well defined graphs demonstrating our different networks, along with designs for how they might be implemented in hardware. We also plan to present graphs demonstrating the effects of different interconnects on cache and overall compute performance using our simulator. 

\subsection{Analysis We Hope to Complete}

Our primary goal is to analyze how different topologies affect the speed of a multi-core workload with communication. We plan to use traces from Professor Railing, along with potentially developing our own traces, and find how different interconnects affect 1. the amount of time the cache is waiting for messages to connect and 2. how much the overall trace speeds up based on the topology being used. 
% Separate your goals into what you PLAN TO ACHIEVE (what you believe you must
% get done to have a successful project and get the grade you expect) and an extra
% goal or two that you HOPE TO ACHIEVE if the project goes really well and you get
% ahead of schedule, as well as goals in case the work goes more slowly. It may not be
% possible to state precise performance goals at this time, but we encourage you be as
% precise as possible. If you do state a goal, give some justification of why you think
% you can achieve it. (e.g., I hope to speed up my starter code 10x, because if I did it
% would run in real-time.)
% • If applicable, describe the demo you plan to show at the poster session (Will it be an
% interactive demo? Will you show an output of the program that is really neat? Will
% you show speedup graphs?). Specifically, what will you show us that will demonstrate
% you did a good job?
% • If your project is an analysis project, what are you hoping to learn about the workload
% or system being studied? What question(s) do you plan to answer in your analysis?
% • Systems project proposals should describe what the system will be capable of and
% what performance is hoped to be achieved.

\section{Platform Choice}
We are using CADSS because although it does not offer the robustness and research quality results of more complex simulators like Gem 5, it will allow us to quickly start focusing on our problem rather than learning to use the simulator.

We are using SystemVerilog because it is a language we are familiar with and it allows us to easily and directly express the levels of parallelism needed in describing a network. It will also allow us to experiment with synthesis and power consumption. 

\section{Schedule}

\begin{enumerate}
    \item Week 1: Run basic traces on CADSS and integrate with a very simple SystemVerilog network
    \item Week 2: Develop a more complete network in SystemVerilog and test its integration with the simulator
    \item Week 3: Develop 3 basic networks and compare performance. Describe results for midpoint result.
    \item Week 4: Develop a more advanced network or pairs of networks to compare. 
    \item Week 5: Finish any other topologies. Perform synthesis analysis on power and area. Compare performance and develop final report. 
\end{enumerate}


\end{document}
